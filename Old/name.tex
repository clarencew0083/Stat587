Latex Example: Part 1 Answers
Instructions
Here is one way to do the exercises:
Latex Template
\documentclass{article}
\usepackage{graphicx,fancyhdr,amsmath,amssymb,amsthm,subfig,url,hyperref,booktabs, multicol, multirow}
\usepackage[margin=1in]{geometry}
%----------------------- Macros and Definitions --------------------------


\newcommand{\studentname}{First Last}
\newcommand{\suid}{first.last}
\newcommand{\exerciseset}{Homework N}

\renewcommand{\theenumi}{\bf \Alph{enumi}}

%\theoremstyle{plain}
%\newtheorem{theorem}{Theorem}
%\newtheorem{lemma}[theorem]{Lemma}

\fancypagestyle{plain}{}
\pagestyle{fancy}
\fancyhf{}
\fancyhead[RO,LE]{\sffamily\bfseries\large School}
\fancyhead[LO,RE]{\sffamily\bfseries\large OPER ???}
\fancyfoot[LO,RE]{\sffamily\bfseries\large \studentname: \suid @school.edu}
\fancyfoot[RO,LE]{\sffamily\bfseries\thepage}
\renewcommand{\headrulewidth}{1pt}
\renewcommand{\footrulewidth}{1pt}

\graphicspath{{figures/}}

%-------------------------------- Title ----------------------------------

\title{OPER ???: Course \exerciseset}
\author{\studentname}

%--------------------------------- Text ----------------------------------

\begin{document}
\maketitle

\section*{Problem 1: Math}
\begin{enumerate}
\item %A
An in-line equation: $\sum_{i=1}^N{x_i}$


\item %B
A multi-line, aligned set of equations:
\begin{align*}
P(A \vert B) &= \frac{P(A \cap B)}{P(B)} \\
&= \frac{P(B \cap A)}{P(B)} \\
&= \frac{P(B \vert A)P(A)}{P(B)}
\end{align*}


\item %C
An equation with cases:
\begin{align*}
  f(x) &= 
  \begin{cases}
    x & x \geq 0 \\
    -x & x < 0
  \end{cases}
\end{align*}

\end{enumerate}

\section*{Problem 2: Tables}
\begin{enumerate}

\item %A
A simple table made in Latex:

\begin{tabular}{c c}
  x & y \\
  \midrule
  1 & 10 \\
  2 & 20 \\
  3 & 30
\end{tabular}


\item %B 
A more complicated table using Excel2Latex:

% Table generated by Excel2LaTeX from sheet 'Sheet1'
\begin{table}[htbp]
  \centering
  \caption{Excel2Latex Table}
    \begin{tabular}{|c|c|c|}
    \toprule
          & \multicolumn{2}{c|}{\textbf{Output}} \\
    \textbf{n} & \multicolumn{1}{c}{\textbf{x}} & \textbf{y} \\
    \midrule
    1     & 10    & 11 \\
    \midrule
    2     & 20    & 21 \\
    \midrule
    3     & 30    & 31 \\
    \bottomrule
    \end{tabular}%
  \label{tab:addlabel}%
\end{table}%

\end{enumerate}

\end{document}