\documentclass{article}
\usepackage{graphicx,fancyhdr,amsmath,amssymb,amsthm,subfig,url,hyperref,booktabs, multicol, multirow, mathtools, array, setspace}
\usepackage[margin=1in]{geometry}

%----------------------- Macros and Definitions --------------------------


\newcommand{\studentname}{Clarence Williams}
\newcommand{\suid}{clarence.williams}
\newcommand{\exerciseset}{Homework 0}

\renewcommand{\theenumi}{\bf \Alph{enumi}}

%\theoremstyle{plain}
%\newtheorem{theorem}{Theorem}
%\newtheorem{lemma}[theorem]{Lemma}

\fancypagestyle{plain}{}
\pagestyle{fancy}
\fancyhf{}
\fancyhead[RO,LE]{\sffamily\bfseries\large AFIT ENS}
\fancyhead[LO,RE]{\sffamily\bfseries\large MATH 000}
\fancyfoot[LO,RE]{\sffamily\bfseries\large \studentname: \suid @school.edu}
\fancyfoot[RO,LE]{\sffamily\bfseries\thepage}
\renewcommand{\headrulewidth}{1pt}
\renewcommand{\footrulewidth}{1pt}

\graphicspath{{figures/}}

%-------------------------------- Title ----------------------------------

\title{MATH 000: Course \exerciseset}
\author{\studentname}

%--------------------------------- Text ----------------------------------

\begin{document}
\maketitle

\section*{Example 3}
Compute \(\boldsymbol{AB}, where\ \boldsymbol{A} = \begin{bmatrix*}[r]
2 & 3\\
1 & -3
\end{bmatrix*}\ and\ \boldsymbol{B} = \begin{bmatrix*}[r]
4 & 3 &\quad 6\\
1 & -2 &\quad 3
\end{bmatrix*}\)\\
\\

\(\boldsymbol{AB} = \begin{bmatrix*}[r]

2(4) + 3(1) &\quad 2(3) + 3(-2) &\quad 2(6) + 3(3)\\
1(4) - 5(1) &\quad 1(3) -5(-2) &\quad 1(6) -5(3)

\end{bmatrix*}
= \begin{bmatrix*}[r]

11 &\quad 0 &\quad 21\\
-1 &\quad 12 &\quad -3

\end{bmatrix*}\)

\section*{1.1}
\(x_1 + 2x_2 = 3\\
2x_1 + x_2 = 3\\
\\
\begin{bmatrix*}[r]
1 & 2\\
2 & 1
\end{bmatrix*}
\begin{bmatrix*}[r]
x_1 \\
x_2 
\end{bmatrix*} = \begin{bmatrix*}[r]
3\\
3
\end{bmatrix*}\\ \\ \\
\begin{bmatrix*}[r]
1 &\quad 2 &\quad 3\\
2 &\quad 1 &\quad 3
\end{bmatrix*} \sim  \begin{bmatrix*}[r]
-2 & -4 & -6\\
2 & 1 & 3
\end{bmatrix*} \sim  \begin{bmatrix*}[r]
-2 & -4 & -6\\
0 & -3 & -3
\end{bmatrix*} \sim  \begin{bmatrix*}[r]
-2 & -4 & -6\\
0 & 1 & 1
\end{bmatrix*} \sim  \begin{bmatrix*}[r]
-2 &\; 0 &\;-2\\
0 &\; 1 &\; 1
\end{bmatrix*} \sim  \begin{bmatrix*}[r]
1 & 0 & 1\\
0 & 1 & 1
\end{bmatrix*}
\)\\ \\ \\
Thus, \( \begin{bmatrix*}[r]
x_1\\ x_2 \end{bmatrix*} = \begin{bmatrix*}[r] 1\\ 1\end{bmatrix*}
\)
\end{document}\documentclass{article}
\usepackage[utf8]{inputenc}

\title{First}
\author{clarence.williams }
\date{September 2018}

\begin{document}

\maketitle

\section{Introduction}



\end{document}
